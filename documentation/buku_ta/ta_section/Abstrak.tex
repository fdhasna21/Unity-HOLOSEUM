\begin{center}
\Large\textbf{ABSTRAK}
\end{center}
\vspace{1ex}

\begin{adjustwidth}{-0.2cm}{}
\begin{tabular}{lcp{0.6\linewidth}}
	Nama Mahasiswa &:& Fernanda Daymara Hasna \\
	Judul Tugas Akhir &:& Visualisasi Objek 3D menggunakan \textit{Interactive Holographic Projection}\\
	Dosen Pembimbing &:& 1. Dr. Surya Sumpeno, S.T., M.Sc. \\
	& & 2. Ahmad Zaini, S.T., M.Sc.  \\
\end{tabular}
\end{adjustwidth}
\vspace{1ex}

\setlength{\parindent}{0cm} Kemudahan anak-anak dalam mengakses teknologi dapat memberikan dampak negatif seperti kecanduan, pengaksesan konten yang tidak sesuai, hingga masalah kesehatan fisik dan mental\cite{sundus2018impact}. Agar manfaatnya tetap dapat dirasakan maka konten yang diakses diarahkan pada bidang edukasi, salah satunya tentang perkembangan peradaban manusia yang masih dianggap membosankan untuk dipelajari\cite{wirawan_2018}. Museum sebagai sarana dalam mempelajarinya justru tidak layak untuk dikunjungi, dimana 435 dari museum yang tercatat berada dalam kondisi yang memprihatinkan menurut Direktorat Pelestarian Cagar Budaya dan Permuseuman Kemdikbud\cite{kemendikbud_2019}. Maka dari itu, dibuatlah sistem \textit{interactive holographic projection} untuk menyampaikan informasi secara efektif dan interaktif sehingga menarik untuk dipelajari. Metode yang digunakan yaitu objek museum ditampilkan dalam bentuk hologram dan dapat digerakkan oleh pengguna menggunakan sensor pengindera tangan Leap Motion. Hasil pengujian menunjukkan bahwa gestur tangan tangan dapat memberikan respon yang bersesuaian sebesar 90.71\%, dengan 72.00\% berhasil diaktifkan menggunakan salah satu tangan dan 97.50\% menggunakan kedua tangan. Hal ini didukung dengan percobaan langsung oleh responden dengan \textit{success rate} sebesar 87.00\%. Berdasarkan kuesioner terhadap 61 responden, 47.50\% responden sangat setuju bahwa sistem ini dapat membantu pembelajaran perkembangan peradaban manusia dan 68.9\% sangat setuju bahwa sistem ini dapat mendukung perkembangan museum dan pendidikan di Indonesia.
\vspace{2ex}

Kata Kunci : Perkembangan peradaban manusia, Hologram, Interactive holographic projection.
\newpage