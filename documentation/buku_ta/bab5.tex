\chapter{PENUTUP}
\vspace{4ex}

\section{Kesimpulan}
\vspace{1ex}
Dari pelaksanaan dan pengujian sistem yang sudah dilakukan, penulis berhasil mengimplementasikan sistem \textit{interactive holographic projection} untuk materi perkembangan peradaban manusia. Kemudian untuk lebih detail dapat ditarik beberapa kesimpulan sebagai berikut:
\begin{enumerate} [nolistsep]
	\item Efek hologram yang efektif diterapkan pada penelitian ini yaitu efek kombinasi dari warna asli objek dengan ilusi cahaya semi-transparan dengan nilai efektivitas sebesar 91.67\%.
	\item Gestur dapat dikenali dan memberikan respons yang sesuai dengan nilai rata-rata 90.71\%.
	\item Gestur yang melibatkan satu tangan, baik kiri atau kanan, dapat memicu respons dengan nilai rata-rata 72.00\%. Sedangkan gestur yang melibatkan kedua tangan sekaligus memiliki tingkat efisiensi sebesar 97.50\%.
	\item Daya komputasi komputer memengaruhi performansi sistem \textit{interactive holographic projection}, sehingga membutuhkan \textit{server computer} dengan minimal spesifikasi \textit{processor} Intel Core i7-7700HQ, \textit{graphic card} NVIDIA GeForce GTX 1050 dengan  RAM 16 GB.
	\item Sebanyak 87.0\% skenario yang dilakukan responden dapat direspons balik oleh sistem sesuai dengan fitur yang dibangun. 
	\item Teknologi \textit{interactive holographic projection} dapat membantu pembelajaran perkembangan peradaban manusia di Indonesia secara lebih menarik dan mengesankan berdasarkan pendapat dari 61 responden dengan hasil :
		\begin{enumerate}
			\item Sebanyak 42.6\% setuju dan 47.5\% sangat setuju bahwa sistem ini dapat membantu pembelajaran perkembangan peradaban manusia.
			\item Sebanyak 47.5\% setuju dan 34.4\% sangat setuju bahwa sistem ini dapat meningkatkan ketertarikan dalam mempelajari perkembangan peradaban manusia.
			\item Sebanyak 39.3\% setuju dan 36.1\% sangat setuju bahwa sistem ini lebih mengesankan daripada melihat koleksi museum secara langsung.
			\item Sebanyak 37.7\% setuju dan 54.1\% sangat setuju bahwa sistem ini dapat diimplementasikan di museum di Indonesia.
			\item Sebanyak 27.9\% setuju dan 68.9\% sangat setuju bahwa sistem ini dapat mendukung perkembangan museum dan pendidikan di Indonesia.
		\end{enumerate}
\end{enumerate}
\vspace{2ex}

\section{Saran}
\vspace{1ex}
Demi pengembangan lebih lanjut mengenai tugas akhir ini, disarankan beberapa langkah lanjutan sebagai berikut :
\begin{enumerate} [nolistsep]
	\item Jika penelitian ini dikembangkan lebih lanjut, dapat menambahkan interaksi dan mencakup objek yang lebih beragam.
	\item Detektor gestur tangan yang dibangun harus dikurangi tingkat sensitivitasnya agar pengguna dapat lebih fleksibel dalam mengikuti gestur tangan yang bersesuaian.
	%\item Metodologi pada penelitian ini dapat diaplikasikan pada bidang penelitian visualisasi lainnya.
\end{enumerate}
\vspace{2ex}