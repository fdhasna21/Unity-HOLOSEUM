\chapter{PENDAHULUAN}
\pagenumbering{arabic}
\vspace{4ex}

\hspace{\parindent} Penelitian ini di latar belakangi oleh berbagai kondisi yang menjadi acuan. Selain itu juga terdapat beberapa permasalahan yang akan dijawab sebagai luaran dari penelitian.
\vspace{2ex}

\section{Latar Belakang}
\vspace{1ex}
	Kemajuan teknologi dan internet memberikan kemudahan akses terhadap segala macam konten dan manfaat untuk kehidupan sehari-hari, tak terkecuali bagi anak-anak. Sudah dikenalkan sejak dini ini memudahkan anak-anak dalam beradaptasi dan terbiasa berinteraksi dengan \textit{gadget}. Hal ini dapat menyebabkan dampak negatif seperti mengetahui informasi yang belum pantas diketahui, kecanduan terhadap gadget, hingga menimbulkan efek terhadap kesehatan fisik dan mental\cite{sundus2018impact}. Sedangkan mereka tidak mengerti apakah informasi yang diterimanya itu baik atau buruk terhadap dirinya, sehingga orang dewasa lah yang berperan dalam menentukan informasi dan konten apa saja yang dibutuhkan dan pantas untuk diterima anak-anak. Dalam hal ini, penggunaan teknologi dapat memberikan dampak positif selama informasi yang disajikan sesuai untuk diterima oleh anak-anak. Informasi yang cocok dan layak untuk diakses anak-anak adalah mengenai edukasi, salah satunya adalah perkembangan peradaban manusia. 
	
	Konten bidang sejarah ini perlu dikenalkan pada anak-anak karena mempelajari sejarah dapat meningkatkan kemampuan berpikir kritis dan mengolah informasi. Presepsi sosial yang tersebar di masyarakat yang menyebabkan kurangnya minat untuk mempelajarinya dikarenakan sejarah sendiri sering dianggap sebagai pelajaran menghafal yang terkesan membosankan\cite{wirawan_2018}. Salah satu metode yang dapat mendukung pembelajaran sejarah adalah melalui kunjungan ke museum terkait. Namun berdasarkan penuturan Direktur Pelestarian Cagar Budaya dan Permuseuman Kementerian Pendidikan dan Kebudayaan (Kemendikbud) Firda Arda Ambas, hampir dari serempat dari 435 museum yang tercatat kondisinya memprihatinkan bahkan termasuk ke dalam kategori tidak layak untuk menyimpan koleksi sejarah dan jarang dikunjungi masyarakat\cite{kemendikbud_2019}. Koleksi sejarah hanya disimpan dan dipajang saja, terkadang kurang terawat sehingga kondisinya mulai rusak dan terkesan kumuh (tidak bagus). Mengunjungi museum dipandang sebagai kegiatan membosankan, sebatas untuk melihat barang-barang yang monoton tanpa adanya suatu hal yang dapat meningkatkan rasa penasaran pengunjung. Padahal sudah seharusnya dan selayaknya fasilitas utama dan pendukung yang disediakan selalu dikembangkan karena media penyampaian informasi yang menarik akan lebih mudah dipahami dan menyenangkan bagi pengunjung (meningkatkan \textit{visitor experience})\cite{sheng2012study}. 
\vspace{2ex} 
 
\section{Permasalahan}
\vspace{1ex}
	Berdasarkan latar belakang yang telah dijelaskan, media informasi berbasis teknologi digital terkait perkembangan peradaban manusia di museum di Indonesia tidak banyak diaplikasikan dan tidak interaktif. Hal ini menyebabkan informasi yang disampaikan kurang maksimal sehingga dibutuhkan sebuah media yang dapat menyampaikan informasi secara efektif dan interaktif.
\vspace{2ex}

\section{Tujuan}
\vspace{1ex}
	Tujuan dari Tugas Akhir ini adalah memberikan alternatif media pembelajaran bidang sejarah perkembangan peradaban manusia dengan memanfaatkan \textit{hologram projector} yang dapat dikontrol oleh \textit{user} sehingga informasi yang disampaikan dapat dipahami lebih mudah dan menyenangkan.
\vspace{2ex}

\section{Batasan Masalah}
\vspace{1ex}
	Untuk memfokuskan permasalahan yang diangkat maka dilakukan pembatasan masalah. Batasan-batasan masalah tersebut diantaranya adalah :
	\vspace{0.5ex}
	\begin{enumerate} [nolistsep]
		\item Dimensi set dan komponen utama yang terdiri dari Leap Motion Controller, Monitor, dan \textit{Pyramid Hologram}.
		\vspace{0.5ex}
		
		\item Objek yang divisualisasikan berupa benda peninggalan perkembangan peradaban manusia  yang diwakilkan dalam zona purbakala, klasik (Hindu-Buddha), kolonial dan pergerakan kemerdekaan serta IPTEK (Ilmu Pengetahuan dan Teknologi).
		\vspace{0.5ex}
				
		\item Fitur yang disajikan berupa gestur untuk mengeksplorasi objek hologram selayaknya di dunia nyata, seperti memutar dan mengubah ukuran objek.
		\vspace{0.5ex}
	\end{enumerate}
	\vspace{2ex}

\section{Sistematika Penulisan}
\vspace{1ex}
	Laporan penelitian tugas akhir ini tersusun dalam sistematika dan terstruktur sehingga mudah dipahami dan dipelajari oleh pembaca maupun seseorang yang ingin melanjutkan perancangan sistem. Sistematika penulisan laporan Tugas Akhir ini yaitu:
	\vspace{0.5ex}
	\begin{enumerate}[nolistsep]
		\item BAB 1 Pendahuluan
		
		Bab I berisi uraian tentang latar belakang, permasalahan, tujuan, batasan masalah, metodologi, dan sistematika penulisan dari penelitian tugas akhir ini.
		\vspace{0.5ex}
	
		\item BAB 2 Tinjauan Pustaka
	
		Bab II berisi tentang uraian secara sistematika teoriteori yang berhubungan dengan permasalahan yang dibahas pada penelitian ini. Teori-teori ini digunakan sebagai dasar dalam penelitian, yaitu teori \textit{Holographic Projection}, perangkat \textit{Leap Motion} dan teori penunjang lain termasuk \textit{datasheet} spesifikasi setiap komponen lain yang membangun sistem.
		\vspace{0.5ex}
	
		\item BAB 3 Perancangan Sistem dan Implementasi
	
		Bab III berisi tentang rancangan pemecahanan masalah beserta implementasinya. Berisikan mengenai perencanaan rancangan, uraian rinci mengenai metodologi yang digunakan, dan pemaparan hasil implementasi sistem yang dilakukan. 
		\vspace{0.5ex}
	
		\item BAB 4 Pengujian dan Analisis
	
		Bab IV berisi tentang pengujian eksperimen yang telah dilakukan terhadap sistem dan parameter yang terlibat. Selain itu, bab ini juga memuat hasil dan analisis dari uji coba yang dilakukan pada Tugas Akhir ini.
		\vspace{0.5ex}
		
		\item BAB 5 Penutup
		
		BAB V berisi tentang kesimpulan yang diambil berdasarkan penelitian dan pengujian yang telah dilakukan. Saran dan kritik yang membangun untuk pengembangan lebih lanjut juga dituliskan pada bab ini.
		\vspace{0.5ex}
	\end{enumerate}
\vspace{2ex}

\begin{comment}
\section{Relevansi}
\vspace{1ex}
	Penelitian Tugas Akhir ini mengenai \colorbox{yellow}{cari tau ini tentang apa isinya, kalau dari PedomanTA isinya Manfaat TA}
\vspace{2ex}
\end{comment}