%% indohyph.tex versi 1.0
%% 24-MEI-2007
%
%  Bentuk Pemenggalan Kata Dalam Bahasa Indonesia
%
%% (c) Copyright 2007 Thodo Timoteus dan Dr. Eri Prasetyo
%
%% Pemenggal kata ini adalah free software (software bebas dan gratis)
%
% Bila anda melakukan perubahan pada file ini,
% tolong ubah nama file-nya agar
% dapat berbeda dengan file yang telah ada ini.
%
% Laporan bug, perkembangan, dan saran dapat anda tujukan kepada:
%
%% Thodo Timoteus
%% Komp. Batan Indah Blok M-45
%% Serpong - Tangerang 15313
%% BANTEN - INDONESIA
%
% thodo_cool@yahoo.com
% thodo_2003@student.gunadarma.ac.id
% http://www.thodo.co.nr/
%
% atau:
%
%% Universitas Gunadarma
%% Kampus D Depok
%% Jl. Margonda Raya - Depok
%% Jawa Barat - Indonesia
%
%%%%%%%%%%%%%%%%%%%%%%%%%%%%%%%%%%%%%%%%%%%%%%%%%%%%%%%%%%%%%%%%
% Bentuk pemenggalan lebih baik digunakan menggunakan
% parameter berikut:
%
 \lefthyphenmin=2
 \righthyphenmin=2
%%%%%%%%%%%%%%%%%%%%%%%%%%%%%%%%%%%%%%%%%%%%%%%%%%%%%%%%%%%%%%%%
%%
\patterns{
a1 e1 i1 o1 u1 % pemenggalan setelah huruf vokal
a2i a2u o2i % pemenggalan tidak dilakukan diantara huruf diftong
2b1d 2b1j 2b1k 2b1n 2b1s 2b1t
2c1k 2c1n
2d1k 2d1n 2d1p
2f1d 2f1k 2f1n 2f1t
2g1g 2g1k 2g1n
2h1k 2h1l 2h1m 2h1n 2h1w
2i1o %% vokal i jika bertemu o dapat dipenggal -> bi-o-lo-gi
2j1k 2j1n
2k1b 2k1k 2k1m 2k1n 2k1r 2k1s 2k1t
2l1b 2l1f 2l1g 2l1h 2l1k 2l1m 2l1n 2l1s 2l1t 2l1q
2m1b 2m1k 2m1l 2m1m 2m1n 2m1p 2m1r 2m1s
2n1c 2n1d 2n1f 2n1j 2n1k 2n1n 2n1p 2n1s 2n1t 2n1v
2p1k 2p1n 2p1p 2p1r 2p1t
2r1b 2r1c 2r1f 2r1g 2r1h 2r1j 2r1k 2r1l 2r1m 2r1n 2r1p 2r1r 2r1s
2r1t 2r1w 2r1y
2s1b 2s1k 2s1l 2s1m 2s1n 2s1p 2s1r 2s1s 2s1t 2s1w
2t1k 2t1l 2t1n 2t1t
2w1t                 % Dua kelompok konsonan
		     % yang akan dipenggal dari
                     % konsonan yang lain
2ng1g 2ng1h 2ng1k 2ng1n 2ng1s    % Tiga kelompok konsonan
2n3s2t % kon-stan-ta
3s4k4ri4p3  % skrip-si  tran-skrip ma-nu-skrip
.me2ng3 .me3ng4a4 .me3ng4i4 .me3ng4u4 .me3ng4e4 .me3ng4o4 %%(baru)
.be2r3 .te2r3 .pe2r3 % prefiks
2ng. 2ny. % jangan memenggal akhiran -ng and -ny di akhir kata
i2o1n % in-ter-na-sio-nal
a2ir % ber-air
1ba1ga2i % se-ba-gai-ma-na
2b1kan. 2c1kan. 2d1kan. 2f1kan. 2g1kan. 2h1kan. 2j1kan. 2l1kan.
2m1kan. 2ng1kan. 2n1kan. 2p1kan. 2r1kan. 2s1kan. 2t1kan. 2v1kan.
2z1kan.  %%akhiran -kan
2b1an. 2c1an. 2d1an. 2f1an. 2g1an. 2h1an. 2j1an. 2l1an.
2m1an. 2ng1an. 2n1an. 2p1an. 2r1an. 2s1an. 2t1an. 2v1an.
2z1an. %% akhiran -an
.a2ta2u % atau-pan
2n1lah. 1lah. %partikel -lah
.ta3ng4an. .le3ng4an. .ja3ng4an. .ma3ng4an. .pa3ng4an. .ri3ng4an.
.de3ng4an.
}
%% Dari aturan di atas terdapat beberapa pengecualian aturan
%% untuk beberapa kata, khususnya yang dimulai dengan awalan
%% ber- dan ter-.
%% Berikut adalah aturannya:
\hyphenation{be-ra-be be-ra-hi be-rak be-ran-da be-ran-dal be-rang
             be-ra-ngas-an be-rang-sang be-ra-ngus be-ra-ni
             be-ran-tak-an be-ran-tam be-ran-tas be-ra-pa be-ras
             be-ren-deng be-re-ngut be-re-rot be-res be-re-wok
             be-ri be-ri-ngas be-ri-sik be-ri-ta be-rok be-ron-dong
             be-ron-tak be-ru-du be-ruk be-run-tun
             peng-eks-por peng-im-por
             te-ra te-rang te-ras te-ra-si te-ra-tai te-ra-wang
	     te-ra-weh te-ri-ak te-ri-gu te-rik te-ri-ma te-ri-pang
	     te-ro-bos te-ro-bos-an te-ro-mol te-rom-pah te-rom-pet
	     te-ro-pong te-ro-wong-an te-ru-buk te-ru-na te-rus
	     te-ru-si pe-rang-kat pe-rin-tah
             }
\endinput 